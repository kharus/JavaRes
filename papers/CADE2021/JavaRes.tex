\documentclass{llncs}

\usepackage{booktabs}
\usepackage{amssymb}
\usepackage{amsfonts}
\usepackage{graphicx}
\usepackage{url}
\usepackage{listings}
\usepackage{hyperref}
\usepackage{wrapfig}

\lstset{language=Python,showstringspaces=false}

%\usepackage{caption}
%\captionsetup[table]{skip=10pt}
%\setlength{\belowcaptionskip}{-10pt}

%\title{Teaching Automated Theorem Proving by Example: PyRes Pedagogical Prover\\(system description)}
\title{JavaRes}

\author{Adam Pease\inst{2}
        \and Stephan Schulz\inst{1}
  }
\institute{
  Articulate Software, USA,
  \email{\tt apease@articulatesoftware.com}
  \and
  DHBW Stuttgart, Germany,
  \email{\tt schulz@eprover.org}
}


\titlerunning{JavaRes - Implementing an Educational Prover, from PyRes}
%\authorrunning{Schulz, Cruanes and Vukmirovic}

\renewcommand{\textfraction}{.01}
\renewcommand{\topfraction}{.99}

\newcommand{\mw}[1]{\ensuremath{\mathit{#1}}}
\newcommand{\nat}{\ensuremath{\mathbb N}}
\newcommand{\integer}{\ensuremath{\mathbb Z}}
\newcommand{\rat}{\ensuremath{\mathbb Q}}
\newcommand{\real}{\ensuremath{\mathbb R}}
\newcommand{\eqn}[2]{\ensuremath{#1\!\simeq\!#2}}
\newcommand{\neqn}[2]{\ensuremath{#1\!\not\simeq\!#2}}
\newcommand{\ueqn}[2]{\ensuremath{#1\dot{\simeq}#2}}

\newcommand{\limpl}{\rightarrow}
\newcommand{\limplies}{\rightarrow} % I mistype this too often ;-)
\newcommand{\ltrue}{\ensuremath{\top}}
\newcommand{\lfalse}{\ensuremath{\bot}}
\newcommand{\lequiv}{\ensuremath{\leftrightarrow}}

\newcommand{\terms}{\ensuremath{\mathit{Term(F,V)}}}
\newcommand{\tops}{\mathop{\mathrm{top}}\nolimits}
\newcommand{\pos}{\mathop{\mathrm{pos}}\nolimits}
\newcommand{\gfpf}{\mathop{\mathrm{gfpf}}\nolimits}
\newcommand{\fpf}{\mathop{\mathrm{fpf}}\nolimits}
\newcommand{\fp}{\mathop{\mathrm{fp}}\nolimits}
\newcommand{\text}[1] {\mbox{#1}}

\renewcommand{\textfraction}{.05}
\renewcommand{\topfraction}{.95}
\renewcommand{\bottomfraction}{.95}

\newcommand{\GInferenzC}[3]
{
\begin{tabular}{c}
  $#1$ \\
  \hline
  \raisebox{-0.4ex}{$#2$} \\
\end{tabular}
  #3
}

\newcommand{\CInferenz}[2]
{
\begin{tabular}{c}
  $#1$ \\
  \hline
  \hline
  \raisebox{-0.4ex}{$#2$} \\
\end{tabular}
}

%\pagestyle{empty}
%\bibliographystyle{alpha}
\bibliographystyle{splncs04}

\begin{document}

\maketitle

\section{Abstract}
Did PyRes~\cite{SP:IJCAR-2020} achieve its goal of being a sufficient
model for learning about how to implement a first-order ATP system?
JavaRes is a demonstration prover patterned after
PyRes~\cite{SP:IJCAR-2020}.  In this paper we discuss in more detail
the architecture and data structures of this prover and the experience
of one of us implementing the prover, without prior expertise
in writing an FOL prover.  We provide performance metrics relative to
PyRes and other well-known provers.  To illustrate the value of
JavaRes for learning about theorem proving we also describe
implementation of an alternate simple clausification algorithm,
graphical proof presentation, implementation of the SInE axiom
selection algorithm, and use with the SUMO theory and syntax.


\section{Introduction}

Automated theorem proving is a fascinating and useful discipline but
can be mystifying for someone not deeply acquainted with the field.
It should be fair to say that most computer science professionals do
not understand the power of inference in FOL and how it provides
distinct capabilities from simpler representations such as graphs or
description logics.  Part of the reason for this lack of general
familiarity may be because the barrier to entry in the field remains
high, despite many decades of work and publication.  Most publications
require a degree of mathematical sophistication to understand, and
even with such capability, a reader will not know which data
structures to use, or which of the many algorithms will be simplest or best to
implement.

We initially began with just an attempt for one of us (Pease) to learn
about automated Theorem Proving (ATP), motivated by decades of work in formal ontology \cite{np01,p11},
and finding that among many excellent books, including \cite{Harrison:HPL-2009},
the first steps to understand ATP were too difficult.  Fortunately, the
creator of the Eprover (Schulz), provided the explanations needed to
understand how and where to begin, and this grew into creation of
PyRes.  To see whether PyRes provided enough structure, we wrote
JavaRes and then used it as a platform to implement several extensions
and see whether both provers could provide a suitable platform for ATP
education.


\section{Programming Language Considerations}

Python is dynamically typed, which makes for flexible and fast implementing, but also makes
for a bit of a mismatch to Java, which is statically typed.  The lack of static
types can also cause confusion, because the intended types are not
documented at all points in the code, since they are determined dynamically at
runtime.  Java may hold an advantage for education by requiring clarity in typing
at compile time.

Java, like Python, also provides call-by-value for simple data types and call-by-reference for
objects.  This is powerful, but also imposes an obligation on the programmer to
be mindful of side-effects in methods.  It does allow for use of strategies such as
shared term rewriting, as in the C++ implementation of Eprover.

As a compiled language, Java also offers a speed advantage, which will be evident
when we show performance results comparing two provers that use the same basic algorithms.

\section{Data Structure}

JavaRes mirrors the data structures of PyRes.  At the bottom of the hierarchy, Terms
are variables (denoted in TPTP syntax as an identifier with an initial capital letter,
as in 'X' or 'Y'), constants (denoted by an initial lowercase as in 'p' or 'partOf'),
functions, which can be nested (as in 'p(X)' and 'g(f(a))'), and sentences, which are predicates and
subterm arguments, like partOf(myWheel,myCar).  Note that once we handle equality, that
is allowed as a predicate symbol.  Next come Literals,
which are possibly negated Terms.  Then come Clauses, which are disjunctions of Literals.
Until this point, the only logical operators considered are equality, negation and disjunction (logical 'or').

All these components have methods for printing themselves, parsing from both TPTP and
SUO-KIF, using the Lexer class, creating a new copy, testing for equality and sort order. Of the
classes mentioned so far, only Term, Literal, and Clause are actually used in proving and Formulas
are parsed and then converted to CNF.  We have Clausifier, which implements the algorithm from
\cite{RN:AI-95} and SmallCNF which implements \cite{NW:SmallCNF-2001}.

To hold a set of Clauses the prover is working on, we also have the class ClauseSet.

\section{Class Structure}

Several classes implement the machinery of proving.  The most fundamental is that of Substitution,
which replaces variable with Terms (including other variables).   The class Unification attempts to
unify the structure of two lists of Terms and results in a set of Substitutions.  Substitutions are
also created when we rename variables in clauses to ensure that variables in different clauses
,which are logically different, also have syntactically different names.  The class Resolution
iterates through Literals in a given pair of Clauses, attempting to find unifiable literals with
opposite signs.  Resolution also handles factoring, in which a Clause is simplified when one of
its Literals unifies with another, thereby enabling removal of the more specific Literal, applying
the resulting Substitution to all remaining Literals.  The class ResControl is simple, finding
all factors within a given Clause, and all resolvants between a given Clause and ClauseSet.  The class
SimpleProofState keeps a set of Clauses that
have already been processed, which is subjected to resolution and factoring, and those Clauses that
have not yet been processed.  Its primary method, saturate(), keeps calling its processClause()
method, picking an arbitrary unprocessed Clause for resolution and factoring, and continuing until
a contradiction is found or there are no more unprocessed clauses to try.  The class Prover1 is
the top level of our simple prover.  It simply calls SimpleProofState reads in a ClauseSet.

\section{Improvements}

We improve on the SimpleProofState class with ProofState. It performs several functions.  In addition
to saturation, it also holds the state of various options and heuristics for
the provers, which we will explain shortly, and the derivation of a proof by contradiction,
consisting of relationships among Clauses that detail how they were derived from the originally
given Clauses.

We also add above Clauses are the class BareFormula, which can have logical operators (such as quantification, implication, equivalence,
and conjunction, as well as less common operators such as '<~>').  We have formula wrappers, in the class Formula,
that include the data seen in TPTP problems, such as the type of the statement ('cnf', 'fof', 'thf' etc),
the formula name and the type of the formula ('conjecture', 'plain' etc) as well as other data
that is determined and used during ATP. In order to convert Formulas
into Clauses, we have the classes Clausify and SmallCNFization, which implement two different algorithm
for conversion to conjunctive normal form.

We implement the class Prover2 in order to support reading TPTP or SUO-KIF formulas and converting them
to CNF before saturation.  Prover2 also has routines to handle setting options and returning proofs.  The new
ProofState class uses a topological sort \cite{DBLP:journals/cacm/Kahn62} to create a linearized proof in
TSTP format from the proof graph.  It also outputs dot format to supporting rendering of a proof graph
by GraphViz.  In addition, it employs an algorithm for answer extraction that attempts to search backwards
from a conjecture with an existentially quantified variable to a supporting clause that contains a binding
for that variable.

\subsection{Indexing, Heuristics, and SInE}


\section{Testing and Examples}

As with TPTP itself, it is very valuable to have many examples.  When examples are implemented as
unit tests they catch bugs as well as explain how the algorithms are supposed to work at each stage.
The example provide a sense of purpose for each class, and are as valuable as the algorithms
themselves.  Often, just having a clear set of examples is sufficient to code at least some version
of the algorithm needed.  Adding more tests and examples has been a key benefit of the implementation
of JavaRes, as it showed what obscure bugs might appear, or what non-obvious errors might exist.
For example, in an early version of unification, the implementation failed to consider all possible
options,just returning a list of one set of substitutions, rather than all possible substitution.
This caused a bug that didn't appear until much later, but was easily explained with an example and
test that should ensure that no future implementer will move forward with an implementation while
being unaware of this issue, should it arise.  Java tests are implemented in the jUnit framework.
Whereas Python coders add tests at the end of each class, in Java tests  are separated into their
own classes.

One of the lessons from implementing JavaRes is that one can never have too many tests.  A number
of bugs, mostly from typographical errors, were not evident in the unit tests and only appeared
once tested on some of the larger problems in the TPTP.  Even smaller TPTP problems could lead to
false confidence, since several successful paths to a contradiction often exist and only on
problems that are challenging will the absence of a particular component of proving, such as
forward subsumption, be a critical issue.

StarExec proved to be an essential resource for scaling up testing.
\bibliography{stsbib}
\end{document}

%%% Local Variables:
%%% mode: latex
%%% eval: (tex-pdf-mode)
%%% TeX-master: t
%%% End:
